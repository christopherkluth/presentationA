% Beamer als Dokumentenklasse verwenden
\documentclass{beamer}

% Kodierung dieser Datei angeben
\usepackage[utf8]{inputenc}

\usepackage{amssymb}

% Schönere Schriftart laden
\usepackage[T1]{fontenc}
\usepackage{lmodern}

% Schönere Schriftart für nicht proportionale Schrift laden
\usepackage{courier}

% Deutsche Silbentrennung verwenden
\usepackage[ngerman]{babel}

\usepackage{tikz}
\usetikzlibrary{intersections}
\usetikzlibrary{positioning,shapes.geometric}

% Paket für Darstellung von Quelltext laden
\usepackage{listings}

% Farbunterstützung muss in Beamer nicht explizit geladen werden

% Paket für weitere Symbole laden.
% (Wird von listings benötigt.)
\usepackage{textcomp}

\definecolor{javared}{rgb}{0.6,0,0} % for strings
\definecolor{javagreen}{rgb}{0.25,0.5,0.35} % comments
\definecolor{javapurple}{rgb}{0.5,0,0.35} % keywords
\definecolor{javadocblue}{rgb}{0.25,0.35,0.75} % javadoc

\lstset{language=Java,
	basicstyle=\ttfamily,
	keywordstyle=\color{javapurple}\bfseries,
	stringstyle=\color{javared},
	commentstyle=\color{javagreen},
	morecomment=[s][\color{javadocblue}]{/**}{*/},
	tabsize=4,
	showspaces=false,
	showstringspaces=false}


 \usetheme{Boadilla}
 
% Formeln mit Serifen setzen
\usefonttheme[onlymath]{serif}

% Navigationssynbole ausblenden
\setbeamertemplate{navigation symbols}{}


\title[Beispiel Präsentation]{Beispiele für Präsentationsfolien}
\author[Christopher Kluth, Timon Wolff]{Christopher Kluth \and Timon Wolff}
\date{Winter 2016}

\begin{document}
	\frame{\titlepage}
	
	
\begin{frame}
	Es gibt 5 verschiedene Arten von Folien
	\begin{itemize}
		\item Aufzählung
		\item Textblöcke
		\item Strukturen und Graphen
		\item Tabellen und Matrizen
		\item Abbildung und Diagramme
	\end{itemize}
\end{frame}
	
	
\begin{frame}{Sandhaufensatz}
	\begin{Satz}[Sandhaufensatz]
		Es gibt keine Sandhaufen.
	\end{Satz}
		
		
	\begin{Beweis}
		\begin{enumerate}
			 \item<2-> Ein Sandkorn ist kein Sandhaufen
			 \item<3-> Sandkörner werden durch hinzufügen eins Sandkorns, nicht zu einem Sandhaufen
			 \item Induktiv folgt die Aussage
		\end{enumerate}	 
	\end{Beweis}
\end{frame}
	
	
	
\begin{frame}{Graphen}	
	\begin{tikzpicture}[scale=1,domain=0:5.0]
		\draw (0,0) -- (5.0,0);
		\draw (0,-1) -- (0,3.0);
		\draw[->] (0,0) -- (5.0,0) node[right] {$x$}; 
		\draw[->] (0,0) -- (0,3.0) node[above] {$f(x)$};
			
		\draw[blue] plot (\x,{\x/3}) node[right] {$f(x)=\frac{x}{3}$}; 
		\draw[red] plot (\x,{exp(\x)/50}) node[right] {$f(x)=\frac{e^x}{50}$};
		\draw[green] plot (\x,{sin(\x r)}) node[right] {$\sin \alpha$};
	\end{tikzpicture}	
\end{frame}
	
	
\begin{frame}{Eignung von Folientypen}
	\begin{tabular}{llllll}
		 & \textbf{\rotatebox{60}{Verständnis}} & \textbf{\rotatebox{60}{Entscheidung}} &
		\textbf{\rotatebox{60}{Struktur}} & \textbf{\rotatebox{60}{Abstraktion}} & \textbf{\rotatebox{60}{Vergleich}} \\
		\hline
		\textbf{Aufzählung} & \checkmark & & \checkmark &  & \\ \hline
		\textbf{Abbildung}  & \checkmark & &  & \checkmark & \\ \hline
		\textbf{Strukturen} &  & \checkmark &  & \checkmark & \checkmark \\ \hline
		\textbf{Tabellen} &  & \checkmark &  &  & \checkmark \\ \hline			\textbf{Diagramme} & \checkmark & \checkmark & \checkmark &\checkmark  & \\
		\hline
	\end{tabular}
\end{frame}
	
	
	
\begin{frame}[fragile]{Größter gemeinsamer Teiler}
	 \begin{columns} 
	 	\begin{column}{5cm} 
	 		 \begin{tikzpicture}[
	 		 io/.style={trapezium, trapezium left angle=70, trapezium right angle=110, fill=green!10, draw=green},
			 op/.style={rectangle, fill=blue!10, draw=blue},
			 cn/.style={diamond, aspect=2, inner sep=2pt, fill=purple!10, draw=purple}]
			 
			 \node[io] (in) {Eingabe $a,b$}; 
			 \node[op, below=of in] (div) {$s=a \mod b$}; 
			 \node[op, below=of div] (set) {$a=b,\ b=s$}; 
			 \node[cn, below=of set] (cond) {$b=0?$}; 
			 \node[io, below=of cond] (out) {Ausgabe $a$};
				 
			 \path[->] 
			 (in) edge (div) 
			 (div) edge (set) 
			 (set) edge (cond) 
			 (cond) edge node [right]{Ja} (out);
			 \draw[->]
			 (cond) -- ++(1.5,0) node[below]{Nein}
			 |- (div);
	 		 \end{tikzpicture}
		 		
	 	\end{column} 
		 	
		 	
		 	
	 	\begin{column}{5cm} 
	 		\begin{lstlisting}[gobble=4,language=Java]
	function ggt(a,b){
		var s;
		while(b!=0){
			s=a%b;
			a=b;
			b=s;
		}
		return a;
	}
	 		\end{lstlisting}
		  \end{column} 		 
		\end{columns}	 
\end{frame}
\end{document}